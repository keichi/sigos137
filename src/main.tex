\documentclass[submit,techrep,noauthor]{ipsj}

\usepackage{algorithm}
\usepackage{algorithmic}
\usepackage{amssymb}
\usepackage{amsmath}
\usepackage[japanese]{babel}
\usepackage{booktabs}
\usepackage{cite}
\usepackage{fancyvrb}
\usepackage[dvipdfmx]{graphicx}
\usepackage[utf8]{inputenc}
\usepackage{listings}
\usepackage{longtable}
\usepackage{newclude}
\usepackage{scrhack}
\usepackage{url}
\usepackage{xcolor}
\RequirePackage[l2tabu, orthodox]{nag}

\lstset{
    aboveskip=1em,
    basicstyle=\ttfamily\lst@ifdisplaystyle\footnotesize\fi,
    keywordstyle=\color[RGB]{33,74,135}\bfseries,
    stringstyle=\color[RGB]{79,153,5},
    commentstyle=\color[RGB]{143,89,2}\itshape,
    numberstyle=\footnotesize,
    numbers=none,
    stepnumber=1,
    numbersep=15pt,
    backgroundcolor=\color[RGB]{251,251,251},
    frame=single,
    frameround=tttt,
    framesep=5pt,
    rulecolor=\color[RGB]{148,150,152},
    breaklines=true,
    breakautoindent=true,
    breakatwhitespace=true,
    breakindent=25pt,
    showspaces=false,
    showstringspaces=false,
    showtabs=false,
    tabsize=2,
    captionpos=b,
    linewidth=\textwidth,
}

\providecommand{\tightlist}{%
  \setlength{\itemsep}{0pt}\setlength{\parskip}{0pt}}

\def\newblock{\hskip .11em plus .33em minus .07em}

\bibliographystyle{ipsjsort}

\graphicspath{{img/}}

\setcounter{巻数}{53}%vol53=2012
\begin{document}

\title{MPI通信パターンに基づくSDN制御を高速化する\\
    カーネルモジュールの試作と評価}

\affiliate{IST}{大阪大学大学院情報科学研究科\\
Graduate School of Information Science and Technology,\\
Osaka University}

\affiliate{CMC}{大阪大学大学サイバーメディアセンター\\
Cybermedia Center, Osaka University}

\author{高橋 慧智}{}{IST}[takahashi.keichi@ais.cmc.osaka-u.ac.jp]
\author{伊達 進}{}{CMC}[date@cmc.osaka-u.ac.jp]
\author{Khureltulga Dashdavaa}{}{IST}[huchka@ais.cmc.osaka-u.ac.jp]
\author{木戸 善之}{}{CMC}[kido@cmc.osaka-u.ac.jp]
\author{下條 真司}{}{CMC}[shimojo@cmc.osaka-u.ac.jp]

\begin{abstract}
われわれは,ネットワーク内のパケットフローを動的に制御可能とするSDNに着眼し
,並列分散プロセス間通信ライブラリMPI の通信性能向上を目的とし,SDN-MPIの
研究開発を推進してきた.
SDN-MPIのプロトタイプ実装を通じて,本研究では,個別のMPI関数を単独で実行した際の通信
時間の短縮を実証した.しかし,これまでの実装は,MPIの個別の集団通
信の高速化を実現するに留まっており,複数の集団通信が駆使された実際のMPIア
プリケーションへの応用のためには,アプリケーションが呼び出すMPI関数相互結
合網の制御を連動・連携して行う仕組みが必要不可欠である.本稿では、そのよう
な問題点に着眼し、われわれが開発してきたMPI通信パターンに基づくSDN制御を高
速化するカーネルモジュールについて報告する.
\end{abstract}

\begin{jkeyword}
Message Passing Interface, Software-Defined Networking, Loadable Kernel Module
\end{jkeyword}

\maketitle

\include*{src/1_intro}
\include*{src/2_background}
\include*{src/3_proposal}
\include*{src/4_evaluation}
\include*{src/5_conclusion}

\bibliography{references.bib}

\end{document}
