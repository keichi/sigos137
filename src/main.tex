\documentclass[submit,techrep,noauthor]{ipsj}

\usepackage{algorithm}
\usepackage{algorithmic}
\usepackage{amssymb}
\usepackage{amsmath}
\usepackage[japanese]{babel}
\usepackage{booktabs}
\usepackage[justification=centering]{caption}
\usepackage{cite}
\usepackage{csquotes}
\usepackage{fancyvrb}
\usepackage{graphicx}
\usepackage[utf8]{inputenc}
\usepackage{listings}
\usepackage{longtable}
\usepackage{newclude}
\usepackage{scrhack}
\usepackage{url}
\usepackage{xcolor}
\RequirePackage[l2tabu, orthodox]{nag}

\lstset{
    aboveskip=1em,
    basicstyle=\ttfamily\lst@ifdisplaystyle\footnotesize\fi,
    keywordstyle=\color[RGB]{33,74,135}\bfseries,
    stringstyle=\color[RGB]{79,153,5},
    commentstyle=\color[RGB]{143,89,2}\itshape,
    numberstyle=\footnotesize,
    numbers=none,
    stepnumber=1,
    numbersep=15pt,
    backgroundcolor=\color[RGB]{251,251,251},
    frame=single,
    frameround=tttt,
    framesep=5pt,
    rulecolor=\color[RGB]{148,150,152},
    breaklines=true,
    breakautoindent=true,
    breakatwhitespace=true,
    breakindent=25pt,
    showspaces=false,
    showstringspaces=false,
    showtabs=false,
    tabsize=2,
    captionpos=b,
    linewidth=\textwidth,
}

\providecommand{\tightlist}{%
  \setlength{\itemsep}{0pt}\setlength{\parskip}{0pt}}

\bibliographystyle{ipsjsort}

\setcounter{巻数}{53}%vol53=2012
\begin{document}

\title{MPI通信パターンに基づくSDN制御を高速化する\\
    カーネルモジュールの試作と評価}

\affiliate{IST}{大阪大学大学院情報科学研究科\\
Graduate School of Information Science and Technology,\\
Osaka University}

\affiliate{CMC}{大阪大学大学サイバーメディアセンター\\
Cybermedia Center, Osaka University}

\author{高橋 慧智}{}{IST}[takahashi.keichi@ais.cmc.osaka-u.ac.jp]
\author{Khureltulga Dashdavaa}{}{IST}[huchka@ais.cmc.osaka-u.ac.jp]
\author{木戸 善之}{}{CMC}[kido@cmc.osaka-u.ac.jp]
\author{伊達 進}{}{CMC}[date@cmc.osaka-u.ac.jp]
\author{下條 真司}{}{CMC}[shimojo@cmc.osaka-u.ac.jp]

\begin{abstract}
SDN-MPIは,ネットワーク内のパケットフローを動的に制御可能にするSDNアーキテクチャを,
並列分散環境で用いられるプロセス間通信ライブラリMPIに応用する試みである.
これまで開発してきたSDN-MPIのプロトタイプは,個別のMPI関数を単独で実行した際の
性能向上を実証した.しかし,実際のアプリケーションは複数の異なるMPI関数を使用
するため,アプリケーションの通信パターンに応じて適切なネットワーク制御を実行
する必要がある.本論文では,MPIが送出する各パケットに通信パターンをエンコード
したタグを付与し,タグに基いてネットワークでパケット制御する手法を提案する.
提案手法におけるパケットのタグ付けを高速化するために,MPIライブラリと
連係するカーネルモジュールを試作し,評価した.
\end{abstract}

\begin{jkeyword}
Message Passing Interface, Software Defined Networking, Loadable Kernel Module
\end{jkeyword}

\maketitle

\include*{src/0_intro}
\include*{src/1_background}
\include*{src/2_proposal}
\include*{src/3_evaluation}
\include*{src/4_conclusion}

\bibliography{references.bib}

\end{document}
