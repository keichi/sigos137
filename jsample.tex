\documentclass[submit,techrep,noauthor]{ipsj}

\usepackage{graphicx}

\setcounter{巻数}{53}%vol53=2012
\begin{document}

\title{MPI通信パターンに基づくSDN制御を高速化する\\
    カーネルモジュールの試作と評価}

\affiliate{IST}{大阪大学大学院情報科学研究科\\
Graduate School of Information Science and Technology,\\
Osaka University}

\affiliate{CMC}{大阪大学大学サイバーメディアセンター\\
Cybermedia Center, Osaka University}

\author{高橋 慧智}{}{IST}[takahashi.keichi@ais.cmc.osaka-u.ac.jp]
\author{Khureltulga Dashdavaa}{}{IST}[huchka@ais.cmc.osaka-u.ac.jp]
\author{木戸 善之}{}{CMC}[kido@cmc.osaka-u.ac.jp]
\author{伊達 進}{}{CMC}[date@cmc.osaka-u.ac.jp]
\author{下條 真司}{}{CMC}[shimojo@cmc.osaka-u.ac.jp]

\begin{abstract}
SDN-MPIは,ネットワーク内のパケットフローを動的に制御可能にするSDNアーキテクチャを,
並列分散環境で用いられるプロセス間通信ライブラリMPIに応用する試みである.
これまで開発してきたSDN-MPIのプロトタイプは,個別のMPI関数を単独で実行した際の
性能向上を実証した.しかし,実際のアプリケーションは複数の異なるMPI関数を使用
するため,アプリケーションの通信パターンに応じて適切なネットワーク制御を実行
する必要がある.本論文では,MPIが送出する各パケットに通信パターンをエンコード
したタグを付与し,タグに基いてネットワークでパケット制御する手法を提案する.
提案手法におけるパケットのタグ付けを高速化するために,MPIライブラリと
連係するカーネルモジュールを試作し,評価した.
\end{abstract}

\begin{jkeyword}
Message Passing Interface, Software Defined Networking, Loadable Kernel Module
\end{jkeyword}

\maketitle
\section{はじめに}

情報処理学会では,基幹論文誌として論文誌ジャーナルの発行を行っている.
現在論文誌ジャーナル編集委員会では,
論文誌ジャーナルの論文掲載時のフォーマットとして
A4縦型2段組を採用している.
また,以前は投稿時と掲載時の形式が異なっていたが,
現在では,
投稿時も掲載時と同様のA4縦型2段組で受け付けることにした.

\end{document}
